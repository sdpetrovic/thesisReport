\chapter{Introduction}  
% Last updated: 4-2-2016 
% Rewritten for final thesis report: 04-01-2017
\label{ch:intro}
\acf{MSR} has been a mission concept that has been proposed many times in the past two decades. Today, a large amount of research into this mission is being carried out. Even though it is not yet an official project proposal, \acf{NASA} \acf{JPL} is currently working on pre-cursor missions to eventually fly an \ac{MSR} mission. The reason \ac{NASA} would like to get the samples back to Earth compared to analysing it directly on Mars is the fact that the laboratories on board the current rovers are limited. Should the samples be returned to Earth however, they can be analysed in any laboratory using all available resources.  To prepare for such a mission, research is being conducted on different aspects of \ac{MSR}, such as the ascent trajectory of the \acf{MAV} responsible for transporting soil samples into a Martian orbit. \Cref{ch:problembackground} explains the reasoning behind focusing on the ascent trajectory of the \ac{MSR} as opposed to many other research topics that were available. In the field of ascent trajectory simulation, numerical integrators such as \acf{RKF} are often used to simulate the trajectory propagation. However, in recent years, another integration method called \acf{TSI} has shown promising improvements compared to \ac{RKF} in the field of both orbit trajectory analysis as well as (re-)entry cases, however \ac{TSI} has not yet been used to simulate an ascent trajectory. This is why \ac{TSI} is the focus of this thesis research and thus the primary research question reads: "Does the \ac{TSI} method show similar performance improvements over \ac{RKF} for the Mars ascent case as was observed for the orbital trajectories and (re-)entry cases?". \Cref{ch:problembackground} also discusses some of the early assumptions, and the required reference systems.

To be able to simulate the acent trajectory of the \ac{MAV}, real-life flight conditions have to be modelled in such a way that the results approximate the real ascent. The models used in this research are described in \Cref{ch:models} and are chosen such that the problem is not too complex but at the same time still represents the real-life scenario well enough to be able to draw meaningful conclusions.

As mentioned, different standard integrators are available to be used in such a problem and a selection of those is described in \Cref{ch:standardIntegrationMethods}. Out of those methods, the best comparison integration methods turned out to be the \ac{RKF} methods. From the available \ac{RKF} methods, \acf{RKF78} was chosen for the final comparison with \ac{TSI} because it showed the best performance in the ascent case. 
Because the \ac{RKF78} method is a standard method that is widely available, the main focus and efforts of this research is put on the \ac{TSI} method. In \Cref{ch:tsi} \ac{TSI} is explained in detail and three different formulations are described: a normal cartesian formulation, a unit vector cartesian formulation and a spherical formulation. Eventually it was opted to use the spherical formulation. 

Both \ac{RKF78} and \ac{TSI} are incorporated into a simulation tool, which is capable of computing the ascent trajectory of the \ac{MAV} given a number of initial conditions. The complete set-up of the simulation tool is described in \Cref{ch:programSimulationTool} and includes both existing software as well as software that had to be developed. 
During and after the development of the simulation tool, every aspect of the program had to be verified and validated. As described in \Cref{ch:verificationandvalidation}, this was done in different steps. Two different simulated ascent cases were used to determine the validity of the outcomes of the simulation program.
The same cases were then used for the comparison between \ac{RKF78} and \ac{TSI} but also to perform a sensitivity analysis on the trajectory. All the results and the corresponding analyses are described in \Cref{ch:results}.
From these analyses a number of conclusions can be drawn which are outlined in \Cref{ch:conclusionsandrecommendations}. During the entire research process, many more questions were formed and many more improvements were envisioned. Unfortunately, not all of these questions could be answered nor all improvements implemented in the scope of this thesis. This is why the same chapter also includes a recommendations section for future research. 
Hopefully this research is able to reduce the knowledge gap that currently exists concerning the performance of \ac{TSI} for an ascent trajectory case.





