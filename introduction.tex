\chapter{Introduction}  % Last updated: 4-2-2016 
\label{ch:intro}
\acf{MSR} has been a mission concept that has been proposed many times in the past two decades. Even today, research into this mission is still being done. And although it is not yet an official project proposal, NASAs \acf{JPL} is currently working on pre-cursor missions to eventually launch an \ac{MSR} mission. To prepare for this, research is being conducted on different aspects of \ac{MSR}, such as the \acf{MAV} responsible for transporting the dirt and soil samples into a Martian orbit and the orbiter which will then transport the samples back to Earth. The current orbiter proposed by \ac{JPL} is a low-thrust orbiter called Mars 2022. Such an \ac{MSR} mission requires precise and optimum (optimised for lowest \ac{GLOM}) trajectories to be able to bring back as many samples as possible. But how does one determine the optimum \ac{MAV} trajectory? Especially when it is combined with the optimum trajectory of the low-thrust orbiter. \\
The proposed research would focus on the combined optimisation of an \ac{MAV} trajectory and the trajectory of the low-thrust Mars 2022 orbiter. Also, one hypothesis is that great mass saving can be made if the orbiter and \ac{MAV} would rendezvous within one single orbital revolution after \ac{MAV} lift-off. Therefore, the question that should be answered is: what is the optimal trajectory solution for the combined trajectory problem of a high-thrust \ac{MAV} and a low-thrust Mars orbiter performing a single-revolution rendezvous in Mars orbit? More information on the proposed topic is provided in \Cref{ch:oversub}. \\
A mission such as \ac{MSR} and the corresponding trajectories can be described in many different reference frames, or \ac{RF}, and the motion of the \ac{MAV} and the orbiter can be modelled in different ways. Therefore it is important to use the proper equations and environmental models. Also, the trajectory has to be determined or rather a prediction will have to be made. This can be done using integration methods. And finally, the optimum will have to be found using an optimisation method. All these different aspects are addressed in this literature study.\\
First however, it is important to determine the knowledge that already exists and the research that has already been performed. Therefore, \Cref{ch:missher} will describe previous sample return missions, low-thrust \ac{s/c} missions, single-revolution rendezvous missions and the research performed in those fields. It will also describe the current \ac{MAV} designs. Then before mathematically representing the problem it is important to understand in what kind of \ac{RF} it has to be described. This will be done in \Cref{ch:reftrans}, followed by the \ac{MAV} ascent and low-thrust Mars 2022 orbiter model descriptions in \Cref{ch:launch,ch:transorb} respectively. Here, both chapters explain the assumptions and corresponding equations for each phase. One important aspect of the \ac{MAV} ascent, which sets it apart from other sample return missions, is that Mars has an atmosphere which cannot be neglected. Accordingly \Cref{ch:mars-atm-mod} describes the different atmospheric models and the trade-off that was performed to decide which model to use in this thesis problem. Then the integration and optimisation are discussed in \Cref{ch:int,ch:optimisation} respectively. In the integrators chapter, different integration methods are described and a selection is made of the integration methods that will be used. The same is done for the different optimisers. All of this information will be used to define the final thesis topic, which is presented in \Cref{ch:findef}. For some of the aspects that will be treated in the final thesis problem, certain software is already available. A summary of this software is provided in \Cref{ch:software}. Finally, a proposed schedule is presented in \Cref{ch:findef}, which shows the work which will have to be performed during the thesis work and the time that will have to be spend on each aspect of it. This literature study will serve as a guideline during the thesis project and provide background information for the final thesis report.

%This document has been written to provide a back-ground and better understanding of the preliminary thesis topic and the involved subjects that shall be used during the thesis work which will be performed at NASA \ac{JPL}. All the information gathered and provided in this report is used to finalise the thesis topic and formulate a final thesis proposal. The thesis work at \ac{JPL} is made possible through the \ac{JVSRP}, which has given the opportunity to do research at \ac{JPL} during a period of 1 year.
%The first few months were used to perform this literature study and the last months will be used to carry out the thesis research. During this entire period, the \ac{JPL} staff and resources have been set available to use for the research as long as it does not involve any restricted information or software. At the start of the research period it was known that the topic would involve a Mars mission. Because of restrictions and regulations it was decided that the thesis topic would include trajectory analysis and optimisation and the \ac{MSR} mission. Eventually, the initial topic was set on \ac{MSR} \ac{MAV} ascent trajectory and Mars 2022 orbiter trajectory combined optimisation to rendezvous in Mars orbit. The road-map towards and a more explicit explanation of the thesis topic will be explaine in \Cref{ch:oversub}. In \Cref{ch:missher} different reference missions and research for each aspect of the thesis topic are presented to get an idea of what has already been done and which missions and what research could be used for reference and validation. It is also very important to have a general idea about different mathematical concepts that will have to be used during the thesis work. For this purpose \Cref{ch:optimization} on optimisation and \Cref{ch:int} on integrators were written and form the basic understanding that is required. In the optimisation chapter, different optimisation methods are described and a selection is made of the optimisation methods that will be used. The same is done for the different integrators. The thesis problem will need to be described using mathematical model to simulate the ascent and orbital conditions of the \ac{MAV} and Mars 2022 orbiter respectively. \Cref{ch:launch} describes the used model, the assumptions and the corresponding equations for the ascent phase. In turn \Cref{ch:transorb} will describe the model, assumptions and equations for the orbiter. In these equations, many different reference frames and reference systems are used. This is why \Cref{ch:reftrans} describes the different required reference frames and systems and more importantly the transformation between these. All the required transformations, which follow from \Cref{ch:launch,ch:transorb}, are written out as well. One very important aspect of the \ac{MAV} ascent which sets it apart from other sample return missions is that Mars has an atmosphere which can not be neglected. This is why \Cref{ch:mars-atm-mod} describes the different models and the trade-off that was performed to decide which model will be best to use in this thesis problem. For some of the aspects that will have to be treated in the thesis problem, certain software is already available that can be used. A summary of this software is provided in \Cref{ch:software}. Finally, all the information gathered is used to produce a final thesis topic and corresponding proposed schedule which. The thesis proposal can be found in \Cref{ch:findef}. This entire document will be used during the thesis project and will serve as a guideline and provide background information for the final thesis report.


