\chapter{Nominal case input values} 				
% File created 21-12-2016
\label{app:appendixF-nominalCaseInputValues}

In this appendix, the input files for the two different cases are given.
\Cref{tab:inputFileDefinition} describes the definition of the input parameters and the units.


%\begin{table}[!ht]
%\begin{center}
%\caption{Description of the input parameters}
%\label{tab:inputFileDefinition}
%\begin{tabular}{|l|p{10cm}|}
%\hline 


\begin{longtable}{|l|p{10cm}|}
\caption{Description of the input parameters}
\label{tab:inputFileDefinition}
\endfirsthead
\endhead
\hline

\textbf{Parameter} & \textbf{Meaning} \\ \hline 
desiredOrbitalAltitude & This is the altitude in km \ac{MOLA} that the \ac{MAV} should reach, so the altitude of the desired orbit. At the same time is is also a cut-off criteria for the simulation.  \\ \hline
desiredInclinationDeg & This is the desired inclination in degrees of the target orbit. \\ \hline
initialAltitude & This is the altitude at which the simulation starts, so the launch altitude for the \ac{MAV} in km \ac{MOLA}. \\ \hline
initialLatitudeDeg & This is the launch latitude on Mars in degrees at the start of the simulation. \\ \hline
initialLongitudeDeg & This is the launch longitude on Mars in degrees at the start of the simulation. \\ \hline
FlightPathAngleDeg & This is the flight-path angle in degrees at the start of the simulation. 90 degrees is straight up and 0 degrees is horizontal. \\ \hline
HeadingAngleDeg & This is the angle in degrees that defines the heading of the \ac{MAV} and in the simulation it is defined 0 degrees when pointing North and 90 degrees when pointing East. \\ \hline
initialGroundVelocity & This is the ground velocity in km/s of the \ac{MAV} at the start of the simulation. Because both \ac{TSI} and \ac{RKF} have to do too many steps if the ground velocity starts at 0 km/s, it is better to give it a small initial velocity.\\ \hline
massMAV & This is the \ac{GLOM} in kg of the \ac{MAV}.\\ \hline
thrust & This is the nominal thrust in kN of the \ac{MAV} engine and is a constant throughout the simulation.\\ \hline
specificImpulse & This is the specific impulse in seconds of the \ac{MAV} engine and is a constant throughout the simulation. \\ \hline
initialBurnTime & This is the burn time in seconds of the first burn from the start of the simulation until the coasting phase.\\ \hline
constantThrustElevationAngle & This is the thrust elevation angle in degrees as defined in the propulsion frame and is a constant throughout the simulation. \\ \hline
constantThrustAzimuthAngle & This is the thrust azimuth angle in degrees as defined in the propulsion frame and is a constant throughout the simulation. \\ \hline
maxOrder & This is the order of the \ac{TSI} simulation run, or $K$. \\ \hline
chosenLocalErrorTolerance & This is the error tolerance for both \ac{TSI} and \ac{RKF} during the actual integration and should not go below 10$^{-15}$. \\ \hline
chosenStepSize & This is the initial step-size in seconds that \ac{TSI} starts with. It is the default first step-size of the step-size class that is updated every time from then on. \\ \hline
setEndTime & This is a cut-off criteria and can be chosen to be a certain amount of seconds at which the simulation should stop unless one of the other cut-off criteria is met first. \\ \hline
RKFinitiatorTime & This is the minimum time in seconds that the initiator step should be run. The initiator step is an \ac{RKF} run with an error tolerance of 10$^{-15}$ and stops when it goes beyond the initiator time. This does not mean that it is cut-off at the initiator time exactly. \\ \hline
rotatingPlanet & If this value is 1, the rotation of Mars is taken into account during the simulation. If it is 0 the simulation is run with a non-rotating Mars such that the inertial frame and the rotating frame are the same. \\ \hline
Gravity & If this value is 1, the Martian gravitational acceleration is taken into account. If it is 0, then this is neglected. \\ \hline
Thrust & If this value is 1, the \ac{MAV} thrust acceleration is taken into account. If it is 0, then this is neglected. \\ \hline
Drag & If this value is 1, the drag acceleration is taken into account. If it is 0, then this is neglected. \\ \hline
comparison & If this value is 1, then the comparison results between \ac{TSI} and \ac{RKF} are computed after each integration. If it is 0, then this is not the case. \\ \hline

\end{longtable}
 
%\end{tabular}
%\end{center}
%\end{table}


\section{Case 1}
\label{sec:case1}

\begin{lstlisting}
Test case from Woolley 2015 (case 10 SSTO)
desiredOrbitalAltitude 			= 390.0
desiredInclinationDeg 			= 45.0
initialAltitude 				= -0.6
initialLatitudeDeg 				= 0.0
initialLongitudeDeg				= 74.5
FlightPathAngleDeg 				= 89.0
HeadingAngleDeg 				= 90.0
initialGroundVelocity 			= 0.00001
massMAV 						= 267.4
thrust 							= 3.56
specificImpulse 				= 256
initialBurnTime 				= 142.5
constantThrustElevationAngle	= -0.184
constantThrustAzimuthAngle		= -0.299
maxOrder 						= 20
chosenLocalErrorTolerance 		= 1e-15
chosenStepSize 					= 0.01
setEndTime 						= 2000.0
RKFinitiatorTime 				= 1.0
rotatingPlanet 					= 1
Gravity 						= 1
Thrust 							= 1
Drag 							= 1
comparison 						= 1

\end{lstlisting}

\pagebreak

\section{Case 2}
\label{sec:case2}

\begin{lstlisting}
Test case from Joel (hybrid) case7_3_2016_v33
desiredOrbitalAltitude 			= 479.19000000064
desiredInclinationDeg 			= 92.6899999999988
initialAltitude 				= -0.6
initialLatitudeDeg 				= 0.0
initialLongitudeDeg 			= 90.0
FlightPathAngleDeg 				= 89.0
HeadingAngleDeg 				= 90.0
initialGroundVelocity 			= 0.000001
massMAV 						= 288.95985303149
thrust 							= 6.01868886452604
specificImpulse 				= 315.9
initialBurnTime 				= 99.361911852794
constantThrustElevationAngle	= -0.674
constantThrustAzimuthAngle		= 0.7273
maxOrder 						= 20
chosenLocalErrorTolerance 		= 1e-15
chosenStepSize					= 0.01
setEndTime 						= 2000.0
RKFinitiatorTime 				= 1.0
rotatingPlanet 					= 1
Gravity 						= 1
Thrust 							= 1
Drag 							= 1
comparison 						= 1

\end{lstlisting}

