\chapter*{Preface} 
% Last updated: 4-2-2016
% Updated for final thesis draft: 04-01-2017
\setheader{Preface}

Back in December 2013 Warren Gebbett gave a presentation on his work at the \ac{JPL} in Pasadena, California, USA and the opportunity for a new student to go and perform research at \ac{JPL}. At this point I sent in my application together with eight other students.  Then at the end of December I heard that I was invited for an interview in the first week of January 2014. In this interview it was concluded that I met all the requirements and that I was the perfect candidate to follow Warren up as the next student at \ac{JPL} with financial backing of Dutch Space (now Airbus Defence and Space, the Netherlands). Financial backing was also provided by the Stichting Prof.dr.ir. H.J. van der Maas Fonds (Aerospace Engineering Faculty, TU Delft) and the Stichting Universiteitsfonds Delft (TU Delft). Communication with \ac{JPL} was started in February 2014 and in March 2015 it was clear that I would be working for the Mars Program Formulation Office under the supervision of Roby Wilson (Inner planet mission design group, NASA \ac{JPL}). He told me to focus on subjects that dealt with Mars missions. At that point I was doing my internship at DLR Bremen on Lunar rocket ascent and descent, which lasted till June 2015. When I came back to Delft me and my supervisors Erwin Mooij (rockets, trajectories, entry and descent, TU Delft) and Ron Noomen (mission design and orbit analysis, TU Delft) agreed that it would be best to perform a study on these Mars subjects to prepare for my visit to \ac{JPL} and to formulate proposal thesis topics. The first week at \ac{JPL} I presented these initial thesis topics to both people from the Inner planet mission design group and the Mars program formulation office. The next few weeks were spent choosing and refining one of these topics. And in the months that followed after a literature study was performed to determine the final thesis topic. In the beginning of 2016 I got the green light to start my thesis research and so the journey that would last slightly longer than I expected started. It was not easy, and I quickly realised that I had misjudged the amount of time I would need to actually get my code to work and fix all the mistakes that I made. Completely rewriting models and equations happened so often that I lost count. All of this caused a lot of mental strain which was not always easy to deal with. But with the continuous support of my family, my friends, all my supervisors and colleagues I was still able to keep hope and push on. I never gave up. And even though the research was slower than I expected and a lot of compromised had to be made in the end I am still happy with the results and the amazing opportunity that was given to me. This is why I would personally like to thank my TU Delft supervisors Ron Noomen and Erwin Mooij, for always keeping faith and sticking with me till the end. They were never afraid to speak their mind which always made me stronger. I would also like to thank my supervisor at \ac{JPL}, Roby Wilson for the great support and help during my year long stay in his group. His input always made me critically thing about what I was doing and showed me a slightly different view from the academics that I was used to. \\
During my time abroad I often had some personal struggles as well, which is why I would like to thank my parents, Robert and Linda Petrovi\'{c} for both the financial and mental support that they were able to give me. I would also like to thank the Rutkowski family for hosting me for several periods during my stay in the US and for being my family away from home. \\
Without all of you this thesis research would never have become a reality, which is why I am glad to be able to present the research that I performed the past year in this report. 


\begin{flushright}
{\makeatletter\itshape
    \@author \\
    Edam, The Netherlands, January 2017
\makeatother}
\end{flushright}

