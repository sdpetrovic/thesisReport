\chapter*{Preface} % Last updated: 4-2-2016
\setheader{Preface}

Back in December 2013 Warren Gebbett gave a presentation on his work at the \ac{JPL} in Pasadena, California, USA and the opportunity for a new student to go and perform research at \ac{JPL}. At this point I sent in my application together with eight other students.  Then at the end of December I heard that I was invited for an interview in the first week of January 2014. In this interview it was concluded that I met all the requirements and that I was the perfect candidate to follow Warren up as the next student at \ac{JPL} with financial backing of Dutch Space (now Airbus Defence and Space, the Netherlands). Financial backing was also going to be provided by the Stichting Prof.dr.ir. H.J. van der Maas Fonds (Aerospace Engineering Faculty, TU Delft) and the Stichting Universiteitsfonds Delft (TU Delft). Communication with \ac{JPL} was thus started and in March 2015 it was clear that I would be working for the Mars Program Formulation Office under the supervision of Roby Wilson (Inner Solar System group, NASA \ac{JPL}). He told me to focus on subjects that dealt with Mars missions. At that point I was doing my internship at DLR Bremen on Lunar rocket ascent and descent, which lasted till June 2015. When I came back to Delft me and my supervisors Erwin Mooij (rockets, trajectories, entry and descent, TU Delft) and Ron Noomen (mission design and orbit analysis, TU Delft) agreed that it would be best to perform a study on these Mars subjects to prepare for my visit to \ac{JPL} and to formulate proposal thesis topics. The first week at \ac{JPL} I presented these initial thesis topics to both people from the Inner Solar System group and the Mars program formulation office. The next few weeks were spent choosing and refining one of these topics. This document is the result of the two-month literature study on that topic to prepare for the thesis project.


\begin{flushright}
{\makeatletter\itshape
    \@author \\
    Pasadena, California, February 2016
\makeatother}
\end{flushright}

