\chapter*{Abstract}
% Created: 25-02-2017
\label{ch:Abstract}
Mars Sample Return is a mission concept that has been proposed many times in the past two decades. It is currently still being investigated by different space agencies around the world. The proposal of the Jet Propulsion Laboratory has reached the phase where different elements of this mission concept have been approved and are currently in the design phase. The complete campaign architecture involves three different missions: the first mission will have the objective to collect samples and store them, the second mission will then collect these stored samples and launch them into an orbit around Mars and the third mission will collect the orbiting sample and bring it back to Earth. The first part and the final part are currently secondary objectives for the Mars 2020 and Mars 2022 missions respectively. However, the second mission has not yet reached the approved design phase. 

An important objective of this second mission is to launch the samples into an orbit around Mars. This can be done using a Mars Ascent Vehicle. For such a mission profile, the ascent trajectory has to be simulated during the pre-design phase. An integrator can be used to simulate a Martian ascent by propagating the state of the vehicle in time. Numerical integration methods can be used for such a simulation, however in recent years another integration method has shown potential in both orbital trajectory problems as well as (re)-entry problems. This method is called \acf{TSI} and is based on the propagation of the state using Taylor series. The effectiveness of this integration method for a Mars ascent scenario is tested against the \acf{RKF78} numerical integrator. Therefore the goal of this research is to determine the speed of convergence, the accuracy and the computational speed of both methods. 

Using a set of spherical kinetic and dynamic equations in the rotating frame of Mars, taking into account the gravitational acceleration, the aerodynamic drag and the vehicle thrust, a trajectory was computed utilizing both integrators. The \ac{TSI} method was both more accurate and had a faster speed of convergence, however in this research the computational speed was still slower than the \ac{RKF78} reference numerical integrator. 

\ac{TSI} does, however, show other advantages compared to \ac{RKF78} as well: far fewer function calls, theoretically unlimited order (an optimum of order 20 was determined), can readily handle discontinuities, no interpolation is required afterwards resulting in a theoretically unlimited resolution and it shows less variation in the number of evaluations as a function of the error tolerance. A major disadvantage though of \ac{TSI} is that it can take a long time to set it up since it requires all equations before starting the integration.

An analysis on the influence of different parameters on a Mars ascent trajectory was also analysed using \ac{TSI}. Parameters that influence the trajectory are: launch latitude, launch flight-path angle and launch heading angle.

It can be concluded that \ac{TSI} is a viable integration method for use in ascent problems considering the rapid convergence and the accuracy of the results.

