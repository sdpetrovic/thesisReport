\chapter{All Recurrence Relations}
\label{app:appendixD-allRecurrenceRelations}
In this appendix, all the reduced auxiliary functions that form the complete recurrence relations for each of the variables using the basic recurrence relations are provided for all three cases. Each case contains a table with the function definition, the corresponding equation and the basic recurrence relation category. All the reduced auxiliary derivatives are also described. All these expressions are a function of k, where $W\left(k\right)$ refers to the $k^{th}$ order reduced derivative. The same holds for $X\left(k\right)$ and $U\left(k\right)$. All the expressions hold for $k\geq1$. In the tables, the $\left(k\right)$ is neglected to make it more readable.

\section{First Cartesian case: Euler angles}
\label{sec:firCartApp}
The first Cartesian case requires a large number of reduced auxiliary functions. These are all provided in \Cref{tab:auxFunc1}. For each function, the number, the equations and the category is provided.


\begin{longtable}{|p{1.5cm}|l|p{2cm}|}
\caption{Reduced auxiliary functions for the first Cartesian case}
\label{tab:auxFunc1}
\endfirsthead
\endhead
\hline
\textbf{Auxiliary function} & \textbf{Equation} & \textbf{Category}  \\ \hline \hline
\hline 
$W_{4,1}=$ & $ X_{1}^{2}+X_{2}^{2} $ & Multiplication \\ \hline
$W_{4,2}=$ & $ W_{4,1}+X_{3}^{2} $ & Multiplication \\ \hline
$W_{4,3}=$ & $ \sqrt{W_{4,2}} $ & Power \\ \hline
$W_{4,4}=$ & $ \sqrt{W_{4,1}} $ & Power \\ \hline
$W_{4,5}=$ & $ \dfrac{X_{2}}{W_{4,4}} $ & Division \\ \hline
$W_{4,6}=$ & $ \dfrac{X_{1}}{W_{4,4}} $ & Division \\ \hline
$W_{4,7}=$ & $ \dfrac{X_{3}}{W_{4,3}} $ & Division \\ \hline
$W_{4,8}=$ & $ \dfrac{W_{4,4}}{W_{4,3}} $ & Division \\ \hline
$W_{4,9}=$ & $ X_{4}+\Omega_{M}X_{2} $ & Constant  \\ 
& & Multiplication \\ \hline
$W_{4,10}=$ & $ X_{5}-\Omega_{M}X_{1} $ & Constant  \\ 
& & Multiplication \\ \hline
$W_{4,11}=$ & $ W_{4,9}^{2}+W_{4,10}^{2}+X_{6}^{2} $ & Multiplication \\ \hline
$W_{4,12}=$ & $ \sqrt{W_{4,11}} $ & Power \\ \hline
$W_{4,13}=$ & $ -W_{4,6}W_{4,7} $ & Multiplication \\ \hline
$W_{4,14}=$ & $ -W_{4,7}W_{4,5} $ & Multiplication \\ \hline
$W_{4,15}=$ & $ -W_{4,8}W_{4,6} $ & Multiplication \\ \hline
$W_{4,16}=$ & $ W_{4,8}W_{4,5} $ & Multiplication \\ \hline
$W_{4,17}=$ & $ X_{6}W_{4,8}+W_{4,9}W_{4,13}+W_{4,10}W_{4,14} $ & Multiplication \\ \hline
$W_{4,18}=$ & $ W_{4,10}W_{4,6}-W_{4,9}W_{4,5} $ & Multiplication \\ \hline
$W_{4,19}=$ & $ W_{4,9}W_{4,15}-X_{6}W_{4,7}+W_{4,10}W_{4,16} $ & Multiplication \\ \hline
$W_{4,20}=$ & $ W_{4,17}^{2}+W_{4,18}^{2} $ & Multiplication \\ \hline
$W_{4,21}=$ & $ \sqrt{W_{4,20}} $ & Power \\ \hline
$W_{4,22}=$ & $ \dfrac{W_{4,18}}{W_{4,21}} $ & Division \\ \hline
$W_{4,23}=$ & $ \dfrac{W_{4,17}}{W_{4,21}} $ & Division \\ \hline
$W_{4,24}=$ & $ -\dfrac{W_{4,19}}{W_{4,12}} $ & Division \\ \hline
$W_{4,25}=$ & $ \dfrac{W_{4,21}}{W_{4,12}} $ & Division \\ \hline
$W_{4,26}=$ & $ c\psi_{T} $ & Cosine \\ \hline
$W_{4,27}=$ & $ c\epsilon_{T} $ & Cosine \\ \hline
$W_{4,28}=$ & $ s\psi_{T} $ & Sine \\ \hline
$W_{4,29}=$ & $ s\epsilon_{T} $ & Sine \\ \hline
$W_{4,30}=$ & $ W_{4,26}W_{4,27} $ & Multiplication \\ \hline
$W_{4,31}=$ & $ W_{4,27}W_{4,28} $ & Multiplication \\ \hline
$W_{4,32}=$ & $ \dfrac{1}{X_{7}} $ & Division \\ \hline
$W_{4,33}=$ & $ T W_{4,32} $ & Constant  \\ 
& & Multiplication \\ \hline
$W_{4,34}=$ & $ W_{4,33}W_{4,30} $ & Multiplication \\ \hline
$W_{4,35}=$ & $ \dfrac{X_{27}}{X_{7}} $ & Division \\ \hline
$W_{4,36}=$ & $ W_{4,34}-W_{4,35} $ & Subtraction \\ \hline
$W_{4,37}=$ & $ W_{4,33}W_{4,31} $ & Multiplication \\ \hline
$W_{4,38}=$ & $ W_{4,33}W_{4,29} $ & Multiplication \\ \hline
$W_{4,39}=$ & $ -\mu_{M}\dfrac{X_{1}}{X_{9}} $ & Division \\ \hline
$W_{4,40}=$ & $ -W_{4,7}W_{4,23} $ & Multiplication \\ \hline
$W_{4,41}=$ & $ W_{4,8}W_{4,24} $ & Multiplication \\ \hline
$W_{4,42}=$ & $ -W_{4,5}W_{4,22} $ & Multiplication \\ \hline
$W_{4,43}=$ & $ -W_{4,5}W_{4,23} $ & Multiplication \\ \hline
$W_{4,44}=$ & $ -W_{4,8}W_{4,25} $ & Multiplication \\ \hline
$W_{4,45}=$ & $ W_{4,40}W_{4,25} $ & Multiplication \\ \hline
$W_{4,46}=$ & $ W_{4,42}W_{4,25} $ & Multiplication \\ \hline
$W_{4,47}=$ & $ -W_{4,13}W_{4,22} $ & Multiplication \\ \hline
$W_{4,48}=$ & $ W_{4,40}W_{4,24} $ & Multiplication \\ \hline
$W_{4,49}=$ & $ W_{4,42}W_{4,24} $ & Multiplication \\ \hline
$W_{4,50}=$ & $ W_{4,6}\left(W_{4,45}+W_{4,41} \right) + W_{4,46} $ & Multiplication \\ \hline
$W_{4,51}=$ & $ W_{4,6}\left(W_{4,48}+W_{4,44} \right) + W_{4,49} $ & Multiplication \\ \hline
$W_{4,52}=$ & $ W_{4,39} +W_{4,36}W_{4,50}+W_{4,37}\left(W_{4,47}+W_{4,43}\right)-W_{4,38}W_{4,51} $ & Multiplication \\ \hline
$W_{5,1}=$ & $-\mu_{M}\dfrac{X_{2}}{X_{9}}  $ & Division \\ \hline
$W_{5,2}=$ & $ W_{4,6}W_{4,22} $ & Multiplication \\ \hline
$W_{5,3}=$ & $ W_{4,5} \left(W_{4,45}+W_{4,41}\right) +W_{5,2} W_{4,25} $ & Multiplication \\ \hline
$W_{5,4}=$ & $ -W_{4,14}W_{4,22}+W_{4,6}W_{4,23} $ & Multiplication \\ \hline
$W_{5,5}=$ & $ W_{4,5} \left(W_{4,48}+W_{4,44}\right)+W_{5,2}W_{4,24} $ & Multiplication \\ \hline
$W_{5,6}=$ & $ W_{5,1} + W_{4,36}W_{5,3}+W_{4,37}W_{5,4}-W_{4,38}W_{5,5} $ & Multiplication \\ \hline
$W_{6,1}=$ & $ -\mu_{M}\dfrac{X_{3}}{X_{9}} $ & Division \\ \hline
$W_{6,2}=$ & $ W_{4,7}W_{4,24} $ & Multiplication \\ \hline
$W_{6,3}=$ & $ W_{4,8}W_{4,22} $ & Multiplication \\ \hline
$W_{6,4}=$ & $ -W_{4,7}W_{4,25} $ & Multiplication \\ \hline
$W_{6,5}=$ & $ -W_{4,44}W_{4,23}+W_{6,2} $ & Multiplication \\ \hline
$W_{6,6}=$ & $ W_{4,41}W_{4,23}+W_{6,4} $ & Multiplication \\ \hline
$W_{6,7}=$ & $ W_{6,1} + W_{4,36}W_{6,5}-W_{4,37}W_{6,3}-W_{4,38}W_{6,6} $ & Multiplication \\ \hline
$W_{8,1}=$ & $ X_{1}X_{4} $ & Multiplication \\ \hline
$W_{8,2}=$ & $ X_{2}X_{5} $ & Multiplication \\ \hline
$W_{8,3}=$ & $ X_{3}X_{6} $ & Multiplication \\ \hline
$W_{9,0}=$ & $ X_{9}U_{8} $ & Multiplication \\ \hline
$W_{9,1}=$ & $ \dfrac{W_{9,0}}{X_{8}} $ & Division \\ \hline
$W_{27,1}=$ & $ W_{4,3} $ & Redefining \\ \hline
$W_{27,2}=$ & $ W_{27,1}^{2} $ & Multiplication \\ \hline
$W_{27,3}=$ & $ W_{27,1}^{3} $ & Power \\ \hline
$W_{27,4}=$ & $ W_{27,1}^{4} $ & Power \\ \hline
$W_{27,5}=$ & $ W_{27,1}^{5} $ & Power \\ \hline
$W_{27,6}=$ & $ W_{27,1}^{6} $ & Power \\ \hline
$W_{27,7}=$ & $ W_{27,1}^{7} $ & Power \\ \hline
$W_{27,8}=$ & $ W_{27,1}^{8} $ & Power \\ \hline
$W_{27,9}=$ & $ W_{27,1}^{9} $ & Power \\ \hline
$W_{27,10}=$ & $ W_{27,1}^{10} $ & Power \\ \hline
$W_{27,11}=$ & $ P_{\rho 10}W_{27,10}+P_{\rho 9}W_{27,9}+\dotsc +P_{\rho 1}W_{27,1}+P_{\rho 0} $ & Constant  \\ 
& & Multiplication \\ \hline
$W_{27,12}=$ & $ e^{W_{27,11}} $ & Exponential \\ \hline
$W_{27,13}=$ & $ 
T_{a}=\begin{cases}
P_{T 1,1}W_{27,1}, & \text{for } -0.6 \leq h < 5.04  \\
P_{T 3,2}W_{27,3}+P_{T 2,2}W_{27,2}+P_{T 1,2}W_{27,1}, &  \text{for } 5.04 \leq h < 35.53   \\
P_{T 6,3}W_{27,6}+P_{T 5,3}W_{27,5}+P_{T 4,3}W_{27,4}+ \dotsc   \\
\qquad\ \ \dotsc +P_{T 3,3}W_{27,3}+P_{T 2,3}W_{27,2}+P_{T 1,3}W_{27,1}, &  \text{for } 35.53 \leq h < 75.07   \\
P_{T 8,4}W_{27,8}+P_{T 7,4}W_{27,7}+P_{T 6,4}W_{27,6}+ \dotsc \\
\qquad\ \ \dotsc+P_{T 5,4}W_{27,5}+P_{T 4,4}W_{27,4}+P_{T 3,4}W_{27,3}+ \dotsc \\
\qquad\ \ \dotsc+P_{T 2,4}W_{27,2}+P_{T 1,4}W_{27,1}, &  \text{for } 75.07 \leq h < 170.05   \\
0, &  \text{for }  h \geq 170.05   
\end{cases}
 $ & Constant \mbox{Multiplication} \\ \hline
$W_{27,14}=$ & $ \sqrt{\gamma_{a}R_{a}^{*}W_{27,13}} $ & Power \\ \hline
$W_{27,15}=$ & $ \dfrac{W_{4,12}}{W_{27,14}} $ & Division \\ \hline
$W_{27,16}=$ & $ 
C_{D}=\begin{cases}
0, & \text{for } 0\leq M < 0.5\\
P_{C_{D} 1,2}W_{27,15}, &  \text{for } 0.5\leq M < 1 \\
P_{C_{D} 1,3}W_{27,15}, &  \text{for } 1\leq M < 1.3 \\
P_{C_{D} 1,4}W_{27,15}, &  \text{for } 1.3\leq M < 2.5 \\
P_{C_{D} 1,5}W_{27,15}, &  \text{for } 2.5\leq M < 4 \\
0, &  \text{for } M \geq 4 
\end{cases} $ & Constant \mbox{Multiplication} \\ \hline
$W_{27,17}=$ & $ W_{4,12}^{2} $ & Multiplication \\ \hline
$W_{27,18}=$ & $ W_{27,17}W_{27,16} $ & Multiplication \\ \hline
$W_{27,19}=$ & $ \frac{1}{2} S W_{27,18}W_{27,12} $ & Multiplication \\ \hline

 
% $W_{•,•}=$ & $  $ & \\ \hline
 
\end{longtable}

\noindent
The complete recurrence relation for each of the auxiliary variables can now be written as a function of the reduced auxiliary functions. These are shown by \Cref{eq:allRecRel1} and hold for $k\geq 1$.

\begin{align} \label{eq:allRecRel1}
\begin{split}
U_{1}\left(k\right)&=X_{4}\left(k\right)=\dfrac{U_{4}\left(k-1\right)}{k}\\
U_{2}\left(k\right)&=X_{5}\left(k\right)=\dfrac{U_{5}\left(k-1\right)}{k}\\
U_{3}\left(k\right)&=X_{6}\left(k\right)=\dfrac{U_{6}\left(k-1\right)}{k} \\
\end{split}
&
\begin{split}
U_{4}\left(k\right)&=W_{4,52}\left(k\right)\\
U_{5}\left(k\right)&=W_{5,6}\left(k\right)\\
U_{6}\left(k\right)&=W_{6,7}\left(k\right)\\
\end{split}
&
\begin{split}
U_{7} \left(k\right)&=0 \\
U_{8}\left(k\right)&=2W_{8,1}\left(k\right)+2W_{8,2}\left(k\right)+2W_{8,3}\left(k\right)\\
U_{9}\left(k\right)&=\dfrac{3}{2}W_{9,1}\left(k\right)\\
\end{split}
\end{align}

\noindent
These equations are now all a function of the recurrence relations corresponding to the rest of the auxiliary functions, which are all basic recurrence relations as mentioned in \Cref{tab:auxFunc1}. \\



\section{Second Cartesian case: unit vectors}
\label{sec:secCartApp}
The second Cartesian case is similar to the first one, however fewer reduced auxiliary functions are needed. These are all provided in \Cref{tab:auxFunc2}. For each function, the number, the equations and the category is provided. The numbers are not always the same as in the first case, so these should be read as completely new reduced auxiliary functions. In this case, $u4$, $u5$ and $u6$ are included in the table, because they were defined as recurrence multiplication relations.


\begin{longtable}{|p{1.5cm}|l|p{2cm}|}
\caption{Reduced auxiliary functions for the second Cartesian case}
\label{tab:auxFunc2}
\endfirsthead
\endhead
\hline
\textbf{Function} & \textbf{Equation} & \textbf{Category}  \\ \hline \hline
\hline 
$W_{4,1}=$ & $ X_{1}^{2}+X_{2}^{2} $ & Multiplication \\ \hline
$W_{4,2}=$ & $ W_{4,1}+X_{3}^{2} $ & Multiplication \\ \hline
$W_{4,3}=$ & $ \sqrt{W_{4,2}} $ & Power \\ \hline
$W_{4,4}=$ & $ \sqrt{W_{4,1}} $ & Power \\ \hline
$W_{4,9}=$ & $ X_{4}+\Omega_{M}X_{2} $ & Constant  \\ 
& & Multiplication \\ \hline
$W_{4,10}=$ & $ X_{5}-\Omega_{M}X_{1} $ & Constant  \\ 
& & Multiplication \\ \hline
$W_{4,11}=$ & $ W_{4,9}^{2}+W_{4,10}^{2}+X_{6}^{2} $ & Multiplication \\ \hline
$W_{4,12}=$ & $ \sqrt{W_{4,11}} $ & Power \\ \hline
$W_{4,13}=$ & $ \dfrac{W_{4,9}}{W_{4,12}} $ & Division \\ \hline
$W_{4,14}=$ & $ \dfrac{W_{4,10}}{W_{4,12}} $ & Division \\ \hline
$W_{4,15}=$ & $ \dfrac{X_{6}}{W_{4,12}} $ & Division \\ \hline
$W_{4,16}=$ & $ W_{4,10}X_{3}-X_{6}X_{2} $ & Multiplication \\ \hline
$W_{4,17}=$ & $ X_{6}X_{1}-W_{4,9}X_{3} $ & Multiplication \\ \hline
$W_{4,18}=$ & $ W_{4,9}X_{2}-W_{4,10}X_{1} $ & Multiplication \\ \hline
$W_{4,19}=$ & $ W_{4,16}^{2}+W_{4,17}^{2}+W_{4,18}^{2} $ & Multiplication \\ \hline
$W_{4,20}=$ & $ \sqrt{W_{4,19}} $ & Power \\ \hline
$W_{4,21}=$ & $ \dfrac{W_{4,16}}{W_{4,20}} $ & Division \\ \hline
$W_{4,22}=$ & $ \dfrac{W_{4,17}}{W_{4,20}} $ & Division \\ \hline
$W_{4,23}=$ & $ \dfrac{W_{4,18}}{W_{4,20}} $ & Division \\ \hline
$W_{4,24}=$ & $ W_{4,16}W_{4,23}-W_{4,15}W_{4,22} $ & Multiplication \\ \hline
$W_{4,25}=$ & $ W_{4,15}W_{4,21}-W_{4,13}W_{4,23} $ & Multiplication \\ \hline
$W_{4,26}=$ & $ c\psi_{T} $ & Cosine \\ \hline
$W_{4,27}=$ & $ c\epsilon_{T} $ & Cosine \\ \hline
$W_{4,28}=$ & $ s\psi_{T} $ & Sine \\ \hline
$W_{4,29}=$ & $ s\epsilon_{T} $ & Sine \\ \hline
$W_{4,30}=$ & $ W_{4,26}W_{4,27} $ & Multiplication \\ \hline
$W_{4,31}=$ & $ W_{4,27}W_{4,28} $ & Multiplication \\ \hline
$W_{4,32}=$ & $ \dfrac{1}{X_{7}} $ & Division \\ \hline
$W_{4,33}=$ & $ T W_{4,32} $ & Constant  \\ 
& & Multiplication \\ \hline
$W_{4,34}=$ & $ W_{4,33}W_{4,30} $ & Multiplication \\ \hline
$W_{4,35}=$ & $ \dfrac{X_{27}}{X_{7}} $ & Division \\ \hline
$W_{4,36}=$ & $ W_{4,34}-W_{4,35} $ & Subtraction \\ \hline
$W_{4,37}=$ & $ W_{4,33}W_{4,31} $ & Multiplication \\ \hline
$W_{4,38}=$ & $ W_{4,33}W_{4,29} $ & Multiplication \\ \hline
$W_{4,39}=$ & $ -\mu_{M}\dfrac{X_{1}}{X_{9}} $ & Division \\ \hline
$W_{4,40}=$ & $ W_{4,13}W_{4,22}-W_{4,14}W_{4,21} $ & Multiplication \\ \hline
$U_{4}=$ & $ W_{4,39}+W_{4,36}W_{4,13}+W_{4,37}W_{4,21}-W_{4,38}W_{4,24} $ & Multiplication \\ \hline
$W_{5,1}=$ & $-\mu_{M}\dfrac{X_{2}}{X_{9}}  $ & Division \\ \hline
$U_{5}=$ & $ W_{5,1}+W_{4,36}W_{4,14}+W_{4,37}W_{4,22}-W_{4,38}W_{4,25} $ & Multiplication \\ \hline
$W_{6,1}=$ & $ -\mu_{M}\dfrac{X_{3}}{X_{9}} $ & Division \\ \hline
$U_{6}=$ & $W_{6,1}+W_{4,36}W_{4,15}+W_{4,37}W_{4,23}-W_{4,38}W_{4,40}  $ & Multiplication \\ \hline
$W_{8,1}=$ & $ X_{1}X_{4} $ & Multiplication \\ \hline
$W_{8,2}=$ & $ X_{2}X_{5} $ & Multiplication \\ \hline
$W_{8,3}=$ & $ X_{3}X_{6} $ & Multiplication \\ \hline
$W_{9,0}=$ & $ X_{9}U_{8} $ & Multiplication \\ \hline
$W_{9,1}=$ & $ \dfrac{W_{9,0}}{X_{8}} $ & Division \\ \hline
$W_{27,1}=$ & $ W_{4,3} $ & Redefining \\ \hline
$W_{27,2}=$ & $ W_{27,1}^{2} $ & Multiplication \\ \hline
$W_{27,3}=$ & $ W_{27,1}^{3} $ & Power \\ \hline
$W_{27,4}=$ & $ W_{27,1}^{4} $ & Power \\ \hline
$W_{27,5}=$ & $ W_{27,1}^{5} $ & Power \\ \hline
$W_{27,6}=$ & $ W_{27,1}^{6} $ & Power \\ \hline
$W_{27,7}=$ & $ W_{27,1}^{7} $ & Power \\ \hline
$W_{27,8}=$ & $ W_{27,1}^{8} $ & Power \\ \hline
$W_{27,9}=$ & $ W_{27,1}^{9} $ & Power \\ \hline
$W_{27,10}=$ & $ W_{27,1}^{10} $ & Power \\ \hline
$W_{27,11}=$ & $ P_{\rho 10}W_{27,10}+P_{\rho 9}W_{27,9}+\dotsc +P_{\rho 1}W_{27,1}+P_{\rho 0} $ & Constant  \\ 
& & Multiplication \\ \hline
$W_{27,12}=$ & $ e^{W_{27,11}} $ & Exponential \\ \hline
$W_{27,13}=$ & $ 
T_{a}=\begin{cases}
P_{T 1,1}W_{27,1}, & \text{for } -0.6 \leq h < 5.04  \\
P_{T 3,2}W_{27,3}+P_{T 2,2}W_{27,2}+P_{T 1,2}W_{27,1}, &  \text{for } 5.04 \leq h < 35.53   \\
P_{T 6,3}W_{27,6}+P_{T 5,3}W_{27,5}+P_{T 4,3}W_{27,4}+ \dotsc   \\
\qquad\ \ \dotsc +P_{T 3,3}W_{27,3}+P_{T 2,3}W_{27,2}+P_{T 1,3}W_{27,1}, &  \text{for } 35.53 \leq h < 75.07   \\
P_{T 8,4}W_{27,8}+P_{T 7,4}W_{27,7}+P_{T 6,4}W_{27,6}+ \dotsc \\
\qquad\ \ \dotsc+P_{T 5,4}W_{27,5}+P_{T 4,4}W_{27,4}+P_{T 3,4}W_{27,3}+ \dotsc \\
\qquad\ \ \dotsc+P_{T 2,4}W_{27,2}+P_{T 1,4}W_{27,1}, &  \text{for } 75.07 \leq h < 170.05   \\
0, &  \text{for }  h \geq 170.05   
\end{cases}
 $ & Constant \mbox{Multiplication} \\ \hline
$W_{27,14}=$ & $ \sqrt{\gamma_{a}R_{a}^{*}W_{27,13}} $ & Power \\ \hline
$W_{27,15}=$ & $ \dfrac{W_{4,12}}{W_{27,14}} $ & Division \\ \hline
$W_{27,16}=$ & $ 
C_{D}=\begin{cases}
0, & \text{for } 0\leq M < 0.5\\
P_{C_{D} 1,2}W_{27,15}, &  \text{for } 0.5\leq M < 1 \\
P_{C_{D} 1,3}W_{27,15}, &  \text{for } 1\leq M < 1.3 \\
P_{C_{D} 1,4}W_{27,15}, &  \text{for } 1.3\leq M < 2.5 \\
P_{C_{D} 1,5}W_{27,15}, &  \text{for } 2.5\leq M < 4 \\
0, &  \text{for } M \geq 4 
\end{cases} $ & Constant \mbox{Multiplication} \\ \hline
$W_{27,17}=$ & $ W_{4,12}^{2} $ & Multiplication \\ \hline
$W_{27,18}=$ & $ W_{27,17}W_{27,16} $ & Multiplication \\ \hline
$W_{27,19}=$ & $ \frac{1}{2} S W_{27,18}W_{27,12} $ & Multiplication \\ \hline

 
% $W_{•,•}=$ & $  $ & \\ \hline
 
\end{longtable}

\noindent
The complete recurrence relation for each of the auxiliary variables can again be written as a function of the reduced auxiliary functions. These are shown by \Cref{eq:allRecRel2}.

\begin{align} \label{eq:allRecRel2}
\begin{split}
U_{1}\left(k\right)&=X_{4}\left(k\right)=\dfrac{U_{4}\left(k-1\right)}{k}\\
U_{2}\left(k\right)&=X_{5}\left(k\right)=\dfrac{U_{5}\left(k-1\right)}{k}\\
U_{3}\left(k\right)&=X_{6}\left(k\right)=\dfrac{U_{6}\left(k-1\right)}{k} \\
U_{4}\left(k\right)&=W_{4,39}\left(k\right)+W_{4,36}\left(k\right)W_{4,13}\left(k\right)+W_{4,37}\left(k\right)W_{4,21}\left(k\right)-W_{4,38}\left(k\right)W_{4,24}\left(k\right)\\
U_{5}\left(k\right)&=W_{5,1}\left(k\right)+W_{4,36}\left(k\right)W_{4,14}\left(k\right)+W_{4,37}\left(k\right)W_{4,22}\left(k\right)-W_{4,38}\left(k\right)W_{4,25}\left(k\right)\\
U_{6}\left(k\right)&=W_{6,1}\left(k\right)+W_{4,36}\left(k\right)W_{4,15}\left(k\right)+W_{4,37}\left(k\right)W_{4,23}\left(k\right)-W_{4,38}\left(k\right)W_{4,40}\left(k\right)\\
U_{7} \left(k\right)&=0 \\
U_{8}\left(k\right)&=2W_{8,1}\left(k\right)+2W_{8,2}\left(k\right)+2W_{8,3}\left(k\right)\\
U_{9}\left(k\right)&=\dfrac{3}{2}W_{9,1}\left(k\right)\\
\end{split}
\end{align}



\section{Spherical case}
\label{sec:spherApp}
Since the Spherical case is very different from the Cartesian cases, it comes with its own auxiliary functions. The reduced versions are shown in \Cref{tab:auxFunc3}.

\begin{longtable}{|p{1.5cm}|l|p{2cm}|}
\caption{Reduced auxiliary functions for the Spherical case.}
\label{tab:auxFunc3}
\endfirsthead
\endhead
\hline
\textbf{Auxiliary function} & \textbf{Equation} & \textbf{Category}  \\ \hline \hline
\hline 
$W_{4,4}=$ & $ s X_{12} $ & Sine \\ \hline 
$W_{4,5}=$ & $ c X_{13} $ & Cosine \\ \hline 
$W_{4,6}=$ & $ c X_{12} $ & Cosine \\ \hline 
$W_{4,7}=$ & $ s X_{14} $ & Sine \\ \hline 
$W_{4,8}=$ & $ c X_{14} $ & Cosine \\ \hline 
$W_{4,9}=$ & $ s X_{13} $ & Sine \\ \hline 
$W_{4,26}=$ & $ c \psi_{T} $ & Cosine \\ \hline 
$W_{4,27}=$ & $ c \epsilon_{T} $ & Cosine \\ \hline 
$W_{4,28}=$ & $ s \psi_{T} $ & Sine \\ \hline 
$W_{4,29}=$ & $ s \epsilon_{T} $ & Sine \\ \hline 
$W_{4,32}=$ & $ \dfrac{1}{X_{7}} $ & Division \\ \hline 
$W_{11,0}=$ & $ X_{15}W_{4,8} $ & Multiplication \\ \hline
$W_{11,1}=$ & $ \dfrac{W_{11,0}}{X_{16}} $ & Division \\ \hline
$W_{11,2}=$ & $ W_{11,1}W_{4,9} $ & Multiplication \\ \hline
$W_{11,3}=$ & $ \dfrac{W_{11,2}}{W_{4,6}} $ & Division \\ \hline
$W_{12,1}=$ & $ W_{11,1}W_{4,5} $ & Multiplication \\ \hline
$W_{13,0}=$ & $ W_{4,28}W_{4,27} $ & Multiplication \\ \hline
$W_{13,1}=$ & $ W_{4,7}W_{4,5} $ & Multiplication \\ \hline
$W_{13,2}=$ & $ \Omega_{M}^{2} X_{16} W_{4,6} $ & Multiplication \\ \hline
$W_{13,3}=$ & $ W_{13,2}W_{4,4} $ & Multiplication \\ \hline
$W_{13,4}=$ & $ T W_{13,0} W_{4,32} $ & Multiplication \\ \hline
$W_{13,5}=$ & $ W_{13,3}W_{4,9}+W_{13,4} $ & Multiplication \\ \hline
$W_{13,6}=$ & $ \dfrac{W_{13,5}}{X_{15}} $ & Division \\ \hline
$W_{13,7}=$ & $ -2\Omega_{M}W_{4,6}W_{13,1}+W_{13,6} $ & Multiplication \\ \hline
$W_{13,8}=$ & $ \dfrac{W_{13,7}}{W_{4,8}} $ & Division \\ \hline
$W_{13,9}=$ & $ \left(2\Omega_{M}+W_{11,3} \right)W_{4,4}+W_{13,8} $ & Multiplication \\ \hline
$W_{14,0}=$ & $ T W_{4,29}W_{4,32} $ & Multiplication \\ \hline
$W_{14,1}=$ & $ W_{4,6}W_{4,8}+W_{13,1}W_{4,4} $ & Multiplication \\ \hline
$W_{14,2}=$ & $ -\mu_{M}W_{4,8} $ & Constant \mbox{Multiplication} \\ \hline
$W_{14,3}=$ & $ X_{16}^{2} $ & Multiplication \\ \hline
$W_{14,4}=$ & $ \dfrac{W_{14,2}}{W_{14,3}} $ & Division \\ \hline
$W_{14,5}=$ & $ W_{14,4}+W_{13,2}W_{14,1}+W_{14,0} $ & Multiplication \\ \hline
$W_{14,6}=$ & $ \dfrac{W_{14,5}}{X_{15}} $ & Division \\ \hline
$W_{14,7}=$ & $ 2\Omega_{M}W_{4,6}W_{4,9}+W_{11,1}+W_{14,6} $ & Multiplication \\ \hline
$W_{15,0}=$ & $ TW_{4,26}W_{4,27}-X_{27} $ & Multiplication \\ \hline
$W_{15,1}=$ & $ \dfrac{W_{15,0}}{X_{7}} $ & Division \\ \hline
$W_{15,2}=$ & $ -\mu_{M}W_{4,7} $ & Constant \mbox{Multiplication} \\ \hline
$W_{15,3}=$ & $ \dfrac{W_{15,2}}{W_{14,3}} $ & Division \\ \hline
$W_{15,4}=$ & $ W_{4,8}W_{4,5} $ & Multiplication \\ \hline
$W_{15,5}=$ & $ W_{4,7}W_{4,6}-W_{15,4}W_{4,4} $ & Multiplication \\ \hline
$W_{15,6}=$ & $ W_{13,2}W_{15,5}+W_{15,1}+W_{15,3} $ & Multiplication \\ \hline
$W_{16,1}=$ & $ X_{15}W_{4,7} $ & Multiplication \\ \hline
$W_{27,1}=$ & $ X_{29}U_{28} $ & Multiplication \\ \hline
$W_{27,2}=$ & $ X_{28}U_{29} $ & Multiplication \\ \hline
$W_{27,3}=$ & $ X_{28}X_{29} $ & Multiplication \\ \hline
$W_{27,4}=$ & $ X_{15}\left(W_{27,1}+W_{27,2}\right) $ & Multiplication \\ \hline
$W_{27,5}=$ & $ W_{27,3}U_{15} $ & Multiplication \\ \hline
$W_{27,6}=$ & $ X_{15}\left(W_{27,4}+W_{27,5}\right) $ & Multiplication \\ \hline
$W_{28,1}=$ & $ U_{30}X_{28} $ & Multiplication \\ \hline
$W_{29,1}=$ & $ 
\begin{cases}
0, & \text{for } 0\leq M < 0.5\\
P_{C_{D} 1,2}U_{32}, &  \text{for } 0.5\leq M < 1 \\
P_{C_{D} 1,3}U_{32}, &  \text{for } 1\leq M < 1.3 \\
P_{C_{D} 1,4}U_{32}, &  \text{for } 1.3\leq M < 2.5 \\
P_{C_{D} 1,5}U_{32}, &  \text{for } 2.5\leq M < 4 \\
0, &  \text{for } M \geq 4 
\end{cases}
 $ & Constant \mbox{Multiplication} \\ \hline
$W_{30,1}=$ & $ X_{31}^9 $ & Power \\ \hline
$W_{30,2}=$ & $ X_{31}^8 $ & Power \\ \hline
$W_{30,3}=$ & $ X_{31}^7 $ & Power \\ \hline
$W_{30,4}=$ & $ X_{31}^6 $ & Power \\ \hline
$W_{30,5}=$ & $ X_{31}^5 $ & Power \\ \hline
$W_{30,6}=$ & $ X_{31}^4 $ & Power \\ \hline
$W_{30,7}=$ & $ X_{31}^3 $ & Power \\ \hline
$W_{30,8}=$ & $ X_{31}^2 $ & Multiplication \\ \hline
$W_{30,9}=$ & $ U_{31}\left(10P_{\rho 10}W_{30,1}+\dots +3P_{\rho 3}W_{30,8} + 2P_{\rho 2}X_{31}+P_{\rho 1}\right) $ & Multiplication \\ \hline
$W_{32,1}=$ & $ X_{33}U_{15} $ & Multiplication \\ \hline
$W_{32,2}=$ & $ X_{15}U_{33} $ & Multiplication \\ \hline
$W_{32,3}=$ & $ X_{33}^2 $ & Multiplication \\ \hline
$W_{32,4}=$ & $ \dfrac{W_{32,1}-W_{32,2}}{W_{32,3}} $ & Division \\ \hline
$W_{33,1}=$ & $ \dfrac{U_{34}}{X_{33}} $ & Division \\ \hline
$W_{34,1}=$ & $ 
\begin{cases}
P_{T 1,1}U_{31}, & \text{for } -0.6 \leq h < 5.04  \\
\left(3P_{T 3,2}W_{30,8}+2P_{T 2,2}X_{31}\right)U_{31}+P_{T 1,2}U_{31}, &  \text{for } 5.04\leq h < 35.53   \\
\left(6 P_{T 6,3}W_{30,5}+5P_{T 5,3}W_{30,6}+4P_{T 4,3}W_{30,7}+ \dots
\right. \\
\qquad\  \left. \dotsc +3P_{T 3,3}W_{30,8}+2P_{T 2,3}X_{31}\right)U_{31}+P_{T 1,3}U_{31}, &  \text{for } 35.53\leq h < 75.07   \\
\left(8 P_{T 8,4}W_{30,3}+7P_{T 7,4}W_{30,4}+6P_{T 6,4}W_{30,5}
+5P_{T 5,4}W_{30,6}+ \dots \right. \\
\qquad\  \left. \dotsc +4P_{T 4,4}W_{30,7}+3P_{T 3,4}W_{30,8}+2P_{T 2,4}X_{31}\right)U_{31}+P_{T 1,4}U_{31}, &  \text{for } 75.07\leq h < 170.05   \\
0, &  \text{for }  h \geq 170.05   
\end{cases}
 $ & Multiplication \\ \hline


% $W_{•,•}=$ & $  $ & \\ \hline 
\end{longtable}



\noindent
In this case, a number of extra auxiliary equations were used, which result in extra recurrence relations. These are all described in \Cref{eq:allRecRel3} and hold for $k\geq 1$.

\begin{align} \label{eq:allRecRel3}
\begin{split}
U_{7} \left(k\right)&=0 \\
U_{11}\left(k\right)&=W_{11,3}\left(k\right)\\
U_{12}\left(k\right)&=W_{12,1}\left(k\right)\\
U_{13}\left(k\right)&=W_{13,9}\left(k\right)\\
U_{14}\left(k\right)&=W_{14,7}\left(k\right)\\
U_{15}\left(k\right)&=W_{15,6}\left(k\right)\\
U_{16}\left(k\right)&=W_{16,1}\left(k\right)\\
\end{split}
&
\begin{split}
U_{27}\left(k\right)&=\frac{1}{2}SW_{27,6}\left(k\right)\\
U_{28}\left(k\right)&=W_{28,1}\left(k\right)\\
U_{29}\left(k\right)&=W_{29,1}\left(k\right)\\
U_{30}\left(k\right)&=W_{30,9}\left(k\right)\\
U_{31}\left(k\right)&=U_{16}\left(k\right)\\
U_{32}\left(k\right)&=W_{32,4}\left(k\right)\\
U_{33}\left(k\right)&=W_{33,1}\left(k\right)\\
U_{34}\left(k\right)&=W_{34,1}\left(k\right)\\
\end{split}
\end{align}




%Using the basic recurrence relations, the auxiliary function provided in \Cref{tab:auxFunc} can be written to form the required recurrence relations as is shown by \Cref{eq:sampleRecRel}. Here a sample recurrence relation is provided for each of the basic relations as well as the recurrence relation for $w_{9}$ since this involves both a multiplication as well as a division.
%
%
%\begin{equation} \label{eq:sampleRecRel}
%\begin{split}
%W_{8,1}\left(k\right)=x_{1}x_{4}&=\displaystyle\sum_{j=0}^{k}X_{1}\left(j\right)X_{4}\left(k-j\right)=x_{1}\dfrac{U_{4}\left(k-1\right)}{k}+x_{4}\dfrac{U_{1}\left(k-1\right)}{k}+\displaystyle\sum_{j=1}^{k-1}\dfrac{U_{1}\left(j-1\right)}{j}\dfrac{U_{4}\left(k-j-1\right)}{k-j}\\
%W_{4,1}\left(k\right)=\dfrac{x_{1}}{x_{9}}&=\dfrac{1}{x_{9}}\left[X_{1}\left(k\right)-\displaystyle\sum_{j=1}^{k}X_{9}\left(j\right)W_{4,1}\left(k-j\right)\right]=\dfrac{1}{x_{9}}\left[\dfrac{U_{1}\left(k-1\right)}{k}-\displaystyle\sum_{j=1}^{k}\dfrac{U_{9}\left(j-1\right)}{j}W_{4,1}\left(k-j\right)\right]\\
%W_{14,1}\left(k\right)=x_{16}^{2}&= \dfrac{1}{kx_{16}} \displaystyle\sum_{j=0}^{k-1}\left[2k-j\left(2+1\right)\right] X_{16}\left(k-j\right)W_{14,1}\left(j\right)\\
%&=\dfrac{1}{kx_{16}} \displaystyle\sum_{j=0}^{k-1}\left[2k-j\left(2+1\right)\right] \dfrac{U_{16}\left(k-j-1\right)}{k-j} W_{14,1}\left(j\right) \\
%X_{28}\left(k\right)=\dfrac{U_{28}\left(k-1\right)}{k}=e^{x_{30}}&= \dfrac{1}{k}\displaystyle\sum_{j=0}^{k-1}\left(k-j\right)X_{28}\left(j\right)X_{30}\left(k-j\right)=\dfrac{1}{k}\displaystyle\sum_{j=0}^{k-1}\left(k-j\right)\dfrac{U_{28}\left(j-1\right)}{j}\dfrac{U_{30}\left(k-j-1\right)}{k-j}\\
%&=x_{28}\dfrac{U_{30}\left(k-1\right)}{k}+\dfrac{1}{k}\displaystyle\sum_{j=1}^{k-1}\left(k-j\right)\dfrac{U_{28}\left(j-1\right)}{j}\dfrac{U_{30}\left(k-j-1\right)}{k-j}\\
%W_{4,6}\left(k\right)=\cos x_{12}&= -\dfrac{1}{k}\displaystyle\sum_{j=0}^{k}jW_{4,4}\left(k-j\right)X_{12}\left(j\right)= -\dfrac{1}{k}\displaystyle\sum_{j=1}^{k}jW_{4,4}\left(k-j\right)\dfrac{U_{12}\left(j-1\right)}{j}\\
%W_{4,4}\left(k\right)=\sin x_{12}&= \dfrac{1}{k}\displaystyle\sum_{j=0}^{k}jW_{4,6}\left(k-j\right)X_{12}\left(j\right)= \dfrac{1}{k}\displaystyle\sum_{j=1}^{k}jW_{4,6}\left(k-j\right)\dfrac{U_{12}\left(j-1\right)}{j}\\
%W_{9}\left(k\right)=\dfrac{x_{9}u_{8}}{x_{8}}&=\dfrac{1}{x_{8}}\left[\displaystyle\sum_{j=0}^{k}X_{9}\left(j\right)U_{8}\left(k-j\right)-\displaystyle\sum_{j=1}^{k}X_{8}\left(j\right)W_{9}\left(k-j\right)\right]\\
%&=\dfrac{1}{x_{8}}\left[x_{9}U_{8}\left(k\right)+\displaystyle\sum_{j=1}^{k}\dfrac{U_{9}\left(j-1\right)}{j}U_{8}\left(k-j\right)-\displaystyle\sum_{j=1}^{k}\dfrac{U_{8}\left(j-1\right)}{j}W_{9}\left(k-j\right)\right]\\
%\end{split}
%\end{equation}
%
%Notice how all $X_{n}\left(k\right)$ have been replaced by $\dfrac{U_{n}\left(k-1\right)}{k}$. This way, all recurrence relations are a function of the previous recurrence relations and the initial conditions only.