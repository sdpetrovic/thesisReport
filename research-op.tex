\chapter{Research opportunities}
\label{ch:research}
Throughout this literature study there have been mentions of where the current research done could be or has to be improved. This can either be based on mission requirements that have to be fulfilled or new methods that have not been thoroughly investigated yet. In this chapter, different research opportunities per proposed subject will be given. 


\section{Mars Sample Return research}
\label{sec:msrreop}
In \Cref{ch:optimization} it was already concluded that Differential Evolution and Monotonic Basin Hopping would be the most interesting choice when it comes to optimization of trajectories. Given the different possible ascent and Mars-Earth return trajectories, it might be interesting to perform a comparison of both these methods in these two trajectory cases for a Mars Sample Return mission. It would in the end provide an optimal solution as a result of both these methods. However, considering the different possibilities for return trajectories, a combination of the different methods mentioned in \Cref{ch:transorb} could be optimized on itself using (either of) these two optimization methods. This could even be done by choosing different propellants and engine/rocket configurations mentioned in \Cref{ch:launch} (maybe combining it with a balloon launch) in order to include the optimization of Mars ascent trajectories. Or, if done separately, design the optimum \ac{MAV} with the corresponding ascent trajectory and final orbital conditions. In the case of a long duration return flight (for instance when electric propulsion is used for the return journey back to Earth) the trajectory can be integrated using St\"{o}rmer-Cowell as was concluded in \Cref{ch:int}. Again a comparison could be made between the different integration methods for a low thrust trajectory back to Earth. 
St\"{o}rmer-Cowell can also be used for the integration of the ascent trajectories and different combination of return trajectories. 


\section{Mars Entry and Descent research}
\label{sec:medlreop}
According to \cite{sostaric2010}, with the current \ac{EDL} technologies, only 50$\%$ of the Martian surface is accessible through direct landing (also see \Cref{ch:entry}). And if the landing masses increase, it won't be possible to use the current technology to soft land on Mars. Therefore new technologies and strategies have to be investigated. A lot of research has already been performed on new entry methods and aerodynamic decelerators and even a combination of these has been investigated to create different \ac{EDL} architectures, but mostly for generic missions. Therefore it could be useful to look at the optimal \ac{EDL} sequence specifically for the \ac{MSL} 2 rover or even for a specified manned mission to Mars. Using the same optimization methods mentioned in \Cref{sec:msrreop} an \ac{EDL} trajectory can be optimized to find out what it will take in order to be able to land (almost) everywhere on Mars. Another possibility would be, either in combination with this trajectory optimization or not, to optimize the heat shield and other decelerators for different entries. Thus a feasibility study of the different \acs{IAD}s could be done as well, in order to determine which one would be best suited for what kind of mission. In that case a differentiation can be made between hypersonic \acs{IAD}s and supersonic \acs{IAD}s, both of which have high priority when it comes to development according to the Mars Program Planning Group \cite{mppg2012}. They also mention that the development of high-thrust liquid supersonic retro-propulsion systems should be investigated and could be seen as a stand-alone option or might be used in combination with other decelerators.    

\section{Hazard avoidance research}
\label{sec:hazavreop}
Finally, when it comes to hazard avoidance, there are two possible applications and thus main research areas. An improved hazard avoidance system could be investigated for use on rovers and automated manoeuvring on the Martian surface through optimized paths, or a hazard avoidance system for the automated and guided landing of a Mars lander/rover could be interesting to look into which has a high priority according to the Mars Program Planning Group \cite{mppg2012}. Both these systems could be designed for different main requirements such as: lowest propellant/power usage or path/landing area with the least amount of rocks. 
